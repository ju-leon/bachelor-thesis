%% LaTeX2e class for student theses
%% sections/abstract_en.tex
%% 
%% Karlsruhe Institute of Technology
%% Institute for Program Structures and Data Organization
%% Chair for Software Design and Quality (SDQ)
%%
%% Dr.-Ing. Erik Burger
%% burger@kit.edu
%%
%% Version 1.3.5, 2020-06-26

\Abstract

Metal Catalysts are crucial to a variety of applications, from water splitting to CO2 reduction. %TODO: Besser fomulieren
Catalysts surrounding Vaska's complex can be constructed by combining different ligands around an iridium atom.
Depending on the structure of the resulting element, the activation barrier will vary.

Computing the activation barrier is highly complex and therefore very costly. 
Machine learning can bridge the gap and allow for fast prediction of a catalysts activation barrier.
The methods proposed here are capable of performing high accuracy predictions of the activation barrier based on
the 3D shape of the molecule.

Different ways of encoding a catalysts into a format understandable to a neural network are explored.
Using the special structure of iridium catalysts, the features extracted from the molecules are either fully or partly rotationally invariant.
Additionally, the feature generators proposed allow for a semi bi-directional mapping between feature space and chemical space.

A neural network is then trained to predict the activation barrier from these features.
\\

In a last step, the neural network is used to analyze which parts of the catalyst are important to the prediction of the activation barrier.
With this information an intuition on how the chemical structure has to be changed in order to lower the activation barrier can be learned.
