%% LaTeX2e class for student theses
%% sections/abstract_en.tex
%% 
%% Karlsruhe Institute of Technology
%% Institute for Program Structures and Data Organization
%% Chair for Software Design and Quality (SDQ)
%%
%% Dr.-Ing. Erik Burger
%% burger@kit.edu
%%
%% Version 1.3.5, 2020-06-26

\Abstract

Metal Catalysts are crucial to a variety of applications, from water splitting to CO2 reduction.
Catalysts can be created by combining different parts of a molecule around a metal atom.
Depending on the structure of the resulting ligand, the activation barrier will vary.

Computing this activation barrier is highly complex and therefor very costly. 
Machine learning can bridge the gap and allow for fast and easy computation of a ligands activation barrier.
Additionally, using transfer learning approaches, other properties of the molecule can then be computed with limited effort.

In this bachelor's thesis a machine learning model to predicting activation barriers from the 3D structure of a metal catalyst is found. 
Different ways of encoding a catalysts into a format understandable for a neural notwork are proposed.
Using the special structure of catalyst, the features extracted from the molecules are either fully or partly rotationally invariant.
A neural network is then trained to predict the activation barrier from these features.

Using neural network explainers, the model is analyzed to find which part of the ligands structure are responsible for the activation barrier.
This allows for intuition on how a catalyst might have to be changed for the activation barrier to change in a desired direction.
