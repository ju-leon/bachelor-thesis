%% LaTeX2e class for student theses
%% sections/abstract_en.tex
%% 
%% Karlsruhe Institute of Technology
%% Institute for Program Structures and Data Organization
%% Chair for Software Design and Quality (SDQ)
%%
%% Dr.-Ing. Erik Burger
%% burger@kit.edu
%%
%% Version 1.3.5, 2020-06-26

\Abstract

Metal Catalysts are crucial for a variety of applications, from water splitting to CO2 reduction. %TODO: Besser fomulieren
Catalysts derived from Vaska's complex can be constructed by combining different ligands around an iridium atom.
Depending on the structure of the resulting complex, the activation barrier will vary.


Due to the large space of possible ligand combinations and the complexity of quantum chemistry methods, experimental or virtual exploration
of activation energies is impractical.
Machine learning can bridge the gap and allow for fast prediction of a catalysts activation barrier.
The methods proposed here are capable of performing high accuracy predictions of the activation barrier based on
the 3D shape of the molecule.

Two different ways of encoding a catalysts into a format understandable to a neural network are explored.
The first is using elliptic fourier descriptors to generate fully rotationally invariant features.
The second is using a combination of spherical harmonics and radial basis function to generate a representation of the 
molecule.
Both feature generators proposed allow for a semi bi-directional mapping between feature space and chemical space.

A neural network is then trained to predict the activation barrier from these features.
The best neural network achieves an accuracy of $0.53 kcal/mol$ and $r^2 = 0.96$ on a train fraction of 80\%.

In a last step, the neural network is used to analyze which parts of the catalyst are important to the prediction of the activation barrier.
With this information an intuition on how the chemical structure has to be changed in order to lower the activation barrier can be learned.
