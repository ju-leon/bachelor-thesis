%% LaTeX2e class for student theses
%% sections/content.tex
%% 
%% Karlsruhe Institute of Technology
%% Institute for Program Structures and Data Organization
%% Chair for Software Design and Quality (SDQ)
%%
%% Dr.-Ing. Erik Burger
%% burger@kit.edu
%%
%% Version 1.3.5, 2020-06-26

\chapter{Introduction}
\label{ch:Introduction}

%% -------------------
%% | Example content |
%% -------------------

Metal Catalysts are crucial to a variety of applications, from water splitting to CO2 reduction.
% TODO: Elaborate
The metal catalysts used used here are constructed by combining different structures around a central Iridium atom.
This allows for quick generation of thousands of different catalyst molecules.
For all catalysts in the dataset the activation barrier, along with other properties, is then computed.
Overall, a total of 1947 species with different structures are constructed.
\\
Intuitively no rule for the activation barrier can be found. Seemingly small changes in the catalysts shape can have 
a large influence on the catalysts activation barrier.
\\
Computing this activation barrier is highly complex and therefor very costly. 
Also by just computing the activation barrier, no intuition about which parts of the catalyst contribute to increasing or decreasing the activation barrier can learned.
This intuition however is required if the molecular structure should be adapted later to increase or decrease the activation barrier.
\\
The seemingly arbitrary nature in combination with requiring intuition about the origin of the activation barrier is 
reason to use neural network based regression methods.
Neural networks have become the go-to method for high dimensional regression and classification for their ability to adapt well to complex data.
\\
A limitation of neural networks however is their fixed-size input space.
Therefor a representation of the catalyst needs to be found that encodes our molecule into a fixed-size set of features.
These features will then be fed into our neural network for training. 
\\
In previous works, multiple different techniques are proposed to extract features from a catalyst molecule \cite{friederich_dos}.
While these techniques are based on the chemical structure of a molecule, they do not take into account its 3D spacial structure.
The feature extracting methods developed in this bachelor thesis will rely heavily on the 3D structure of molecule.
The idea being that the 3D structure plays an important role in the activation barrier, and encoding the 3D structure will enhance regression accuracy.
Additionally encoding the 3D structure will allow to learn about the space surrounding the central 
Iridium atom to understand the importance of location of the atoms in our molecule.
\\

3D structural encoding however comes with its own set of challenges. 
One more general being neural networks sensitivity to rotation and positioning.
Since a molecules activation barrier does not change depending on its rotation or location in space, 
information about rotation and translation should ideally not be part of the molecules features.
\\
In the case of the metal catalysts, achieving translational invariance is trivial.
Since every catalyst is constructed around exactly one metal atom, the molecule can be centered around this metal atom.

For rotational invariance the problem is more complex.
Every catalyst has a reaction pocket attached to it's cental atom.
This reaction pocket has a fixed position.
With the vector from the center of the Iridium atom to the center of the reaction pocket, two more degrees of freedom can be removed.
For the last degree of freedom, rotations around this vector, there is no natural way to get rid of it.

Here, 2 different approaches to this last degree of freedom are explored.
The first is a rotationally invariant description using fourier coefficients.

The second is using data augmentation to teach the neural network about all possible rotations.
This means the molecule is rotated along the last remaining axis of freedom, and and multiple examples of the same molecule at different rotations are used as training examples for the neural network.
This ideally allows the network to learn the molecules structure independent of it's rotation.


\section{Dataset}

The dataset contains 1947 examples of catalyst molecules.
For each atom of the molecule, the location in space is given.
For each molecule it's activation barrier is known.

\begin{figure}
  \centering
  \includegraphics[width=7cm]{figures/introduction/barrier.png}
  \caption{Distribution of the elements activation barrier. The elements have a mean of $11.970 kcal/mol$ and a standart deviation of $4.33 kcal/mol$.}
  \label{fig:barriers}
\end{figure}

The molecules were generated by combining ligands around a central Iridium atom as illustrated in \autoref{fig:chemspace}.
The activation barrier was then calculated.
Due to this combinatorial approach the dataset could later be increased with relatively low effort \ref{fig:chemspace}.
Together this gives a dataset containing the structure and the associated activation barrier for 1947 Iridium catalysts.

\begin{figure}
  \centering
  \includegraphics[width=10cm]{figures/introduction/chem-space.png}
  \caption{Ligands defining the chemical space associated with Vaska's complex. Reprinted from \cite{friederich_dos}.}
  \label{fig:chemspace}
\end{figure}

When examining the dataset by hand, the connection between a molecules structure and it's activation barrier is not obvious.
Seemingly small changes can have a big influence on it's activation barrier \ref{fig:struct-diff}.

\begin{figure}[!htb]
  \minipage{0.49\textwidth}
  \includegraphics[width=1.0\textwidth]{figures/introduction/ir_tbp_1_dft-pet3_1_dft-py_1_dft-hicn_1_fluoride_1_smi1_1_s_1.png}
  \endminipage\hfill
  \minipage{0.49\textwidth}
  \includegraphics[width=1.0\textwidth]{figures/introduction/ir_tbp_1_dft-pet3_1_dft-py_1_dft-hicn_1_dft-cn_1_smi1_1_s_1.png}
  \endminipage
  \caption{Two elements with seemingly very similar chemical structures. On the left, IrPC6H15NC5H5CNHFH with an activation barrier of $18.0 kcal/mol$, on the right IrPC6H15NC5H5CNHCNH with an activation barrier of $3.6 kcal/mol$.
  The difference is significant considering the standard deviation shown in \ref{fig:barriers}.  }
  \label{fig:struct-diff}
\end{figure}


A rule to guess the activation barrier from the molecules structure can not easily be found.

Every single element in the dataset has consist of a number of atoms.
The number of atoms forming one molecule varies between elements.
Each atom has a unique position, other chemical properties can be associated with the atom, such as atmoic radius, eleoctromagnetivity and many more.
The location and rotation of the elements in the dataset is seemingly arbitrary.
Note that the global location and rotation of the molecule does not influnce it's activation barrier.
Local changes in atomic position however can have effects on the activation barrier and other properties of the molecule.
  

\section{Previous research}

In previous approaches different machine learning methods were used to predict the activation barrier of elements.
However, the features extracted from the molecule did not take into account the spacial structure of the element.
The elements were instead encoded by creating a graph from the chemical structure.
The elements are then grouped by their distance from the metal center.
For each group of elements, different features are computed as sum of pairwise products/differences of their atomic properties(such as electronegativity, atomic number, identity, topology and size) \cite{friederich_dos}.
Note that these features do not contain any information about the 3D location of the atoms.

Using these autocorrelation features, a neural network and gaussian processes and other forms of regression were used to predict the activation barrier.
Neural networks and gaussian processes both were able to predict the activation barrier within an error of $<1 kcal/mol$ for a test split of $10\%$

Other than the obvious disadvantage of not encoding location information, another disadvantage is the lack of interpretability of results.
While these features succeed at predicting reaction barriers, information what part of the molecule is contributing to the prediction is limited.

The feature extractors introduced in this thesis aim to solve this problem by extracting features that allow for a
partly reconstruction of the chemical space and thus knowing which region of the molecule contributes to a prediction.

Feature extraction is a common problem for machine learning methods in chemical spaces.
Multiple approaches have been proposed for general molecule encoding, 
ranging from encoding properties of molecules, such as the Coulomb Matrix encoder encoding electrostatic interaction of atoms \cite{PhysRevLett.108.058301}
to encoding 3D structures of atomic environments \cite{Bart_k_2013}.

3d structural encoders usually create a fully invariant features description.
In the case of SOAP proposed by \citeauthor{Bart_k_2013}, this is achieved by encoding information about the interaction of 
different species rather than encoding the 3d space itself.
Fully invariant 3d descriptors have the advantage of being universally applicable for any kind of molecule, since they have no requirements to the element being encoded.
The disadvantages are that, once the features are generated, a transformation back to 3D space is not possible.
In many applications, this is not a problem since the generated features are used only for prediction of an elements properties.
In our case, the features however should later allow us to interpret the 3D space surrounding the molecule and ideally give an idea on how the molecule can be changed to alter it's properties.

The idea of using a catalysts special structure for this task, and therefor removing some axes of freedom from the feature space, seems to be an unexplored approach.


\section{Objectives}

This work can be grouped into 2 main objectives. 
The first is to find a feature extractor that generates features from a catalyst molecule that ideally rotationally invariant.
The second objective is to train a neural network on these features that predicts the activation barrier.

\subsection{Feature generation}

The features should allow a regressor to make predictions from the euclidean space surrounding the central Iridium atom.
Therefor features have to be of fixed length for all elements in the dataset.
Ideally the number of features is as low as possible, helping the model to identify the relevant features better.
For interpretability of the results, the features should allow for a partial reconstruction of the euclidean space they are encoding.
This means, given the features, it should be possible to approximate the general shape of the molecule.

The global location and rotation of the element should not influence the features or have only limited influence on the features.

\subsection{Regression}

The second step is to predict the activation barrier from these features using an artificial neural network.
The goal was to find a network able to to predict the activation barrier with accuracy similar or better to the machine learning methods proposed by \citeauthor{friederich_dos}.
The networks proposed here achieve higher accuracy's than the best regression methods proposed in \cite{friederich_dos}.


\subsection{Explaining the feature space}
Since the features should be able to allow a inversion back to the 3D space, it's possible to approximate which areas in 3D space are responsible for the prediction.
Using neural networks explainers, in  a last step the regions in 3d space that influence the regression will be analyzed.
This gives an idea on how different atoms in the dataset influence the prediction.
In further work, this might give intuition on how a molecule needs to be changed to alter it's activation barrier in a desired way.