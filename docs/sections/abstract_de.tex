%% LaTeX2e class for student theses
%% sections/abstract_de.tex
%% 
%% Karlsruhe Institute of Technology
%% Institute for Program Structures and Data Organization
%% Chair for Software Design and Quality (SDQ)
%%
%% Dr.-Ing. Erik Burger
%% burger@kit.edu
%%
%% Version 1.3.5, 2020-06-26

\Abstract

Katalysatoren sind Reaktionsbeschleuniger die in verschiedensten Bereichen eingesetzt werden.
Durch Anordnung von chemischen Strukturen um einen zentralen Kern können verschiedenste Katalystoren erzeugt werden.
In Abhängigkeit vvon der Struktur der Katalysatoren ändert sich auch deren Aktivierungsenergie.

Die Berechnung der Aktivierungsenergie ist mit hohem Aufwand verbunden.
Durch Maschinelles Lernen kann dieser Aufwand verringert werden.
Es werden verschiede Modelle verogestellt die die Aktivierungsenergie eines Katalysators 
anhand seiner dreidimensionalen Struktur mit hoher Genauigkeit schätzen können.

Durch diese Modelle lassen sich ausßerdem Rückschlüsse auf die Struktur des Katalystors ziehen,
und damit 