%% LaTeX2e class for student theses
%% sections/abstract_de.tex
%% 
%% Karlsruhe Institute of Technology
%% Institute for Program Structures and Data Organization
%% Chair for Software Design and Quality (SDQ)
%%
%% Dr.-Ing. Erik Burger
%% burger@kit.edu
%%
%% Version 1.3.5, 2020-06-26

\Abstract

Vaska's Komplex beschreibt eine Iridiumverbindung, von der durch einen kombinatorischen Ansatz 
verschiedenste Katalysatoren abgeleitet werden können.
Die Aktivierungsenergie der entstehenden Katalysatoren ist abhängig von ihrer durch
Liganden gegebenen Struktur. 

Aufgrund der vielen möglichen Kombinationen von Liganden und die Komplexität der zugrunde liegenden quantenchemischen 
Abhängigkeiten ist die Berechnung der Aktivierungsenergie sehr komplex und benötigt entsprechend viel Rechenleistung.
%Die Berechnung dieser Aktivierungsenergie ist sehr komplex und benötigt entsprechend viel Rechenleistung.
Durch den Einsatz von Techniken des maschinellen Lernens kann die Aktivierungsenergie mit hoher Genauigkeit vorausgesagt werden,
ohne dass eine konkrete Berechnung der Aktivierungsenergie nötig ist.

In dieser Bachelorarbeit werden verschiedene Methoden die dreidimensionale Struktur 
eines Katalysators in maschinenverständliche Merkmale zu überführen vorgestellt. 
Die Merkmale werden dabei weitestgehend unabhängig von der Rotation des Elements gehalten.

Mithilfe neuronaler Netzwerke wird anhand der Merkmale die Aktivierungsenergie des Katalysators vorhergesagt.
Die neuronalen Netze erreichen Genauigkeiten bei der Vorhersage von $MAE = 0.53kcal/mol$ bei einem Bestimmtheitsmaß von $r^2 = 0.96$ (trainiert auf 80\% der Daten),
die bis heute höchste erreichte Genauigkeit auf diesem Datensatz.

Da die hier vorgestellten Merkmalsextraktoren einen Rückschluss auf die Struktur des Moleküls zulassen, kann in einem letzten Schritt untersucht werden,
welchen Einfluss bestimmte Teile des Elements auf seine Aktivierungsenergie haben.
Dadurch lassen sich Rückschlüsse ziehen, wie der Katalysator geändert werden muss, um seine Aktivierungsenergie zu verringern.

