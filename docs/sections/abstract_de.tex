%% LaTeX2e class for student theses
%% sections/abstract_de.tex
%% 
%% Karlsruhe Institute of Technology
%% Institute for Program Structures and Data Organization
%% Chair for Software Design and Quality (SDQ)
%%
%% Dr.-Ing. Erik Burger
%% burger@kit.edu
%%
%% Version 1.3.5, 2020-06-26

\Abstract

Vaska's Komplex beschreibt eine Iridium Verbindung, mit der durch einen kombinatorischen Ansatz 
verschiedenste Katalysatoren erzeugt werden können.
Die Aktivierungenergie der entstehenden Katalysatoren ist abhängig von ihrer Struktur. 

Die Berechnung dieser Aktivierungsenergie ist sehr komplex und benötig entsprechend viel Rechenleistung.
Durch den Einsatz von Techniken des maschinellen Lernens kann die Aktivierungsenergie mit hoher Genauigkeit vorraus gesagt werden,
ohne das eine konkrete Berechnung der Aktivierungsenergie nötig ist.

In dieser Bachelorarbeit werden verschiedene Methoden die dreidimensionale Struktur 
eines Katalysators in maschinenverständliche Merkmale zu überführen vorgestellt. 
Die Merkmale werden dabei weitesgehend unabhängig von der Rotation des Elements gehalten.

Mit Hilfe neuronaler Netzwerke wird anhand der Merkmale die Aktivierungsenergie des Katalysators vorhergesagt.

Da die hier vorgestellten Merkmalextraktoren einen Rückschluss auf die Struktur des Moleküls zulassen, kann in einem letzten Schritt untersucht werden,
welchen Einfluss bestimmte Teile des Elements auf seine Aktivierungsenergie haben.


